\section{Ход выполнения работы}

Все задания практикума выполнялись по варианту 11.

Устройство формирования индексов SQL EXCEPT. Сформировать в хост-подсистеме и передать в SPE 256 записей множества A (случайные числа в диапазое 0..1024) и 256 записей множества B (случайные числа в диапазоне 0..1024). Сформировать в SPE множество C = A not B. Выполнить тестирование работы SPE, сравнив набор ключей в множестве C с ожидаемым.


В листинге~\ref{lst1} представлен код программы по индивидуальному варианту из файла $host\_main.cpp$. В листинге~\ref{lst2} представлен код программы по индивидуальному варианту из файла $sw\_kernel\_main.c$.

\begin{lstlisting}[caption={Код программы по индивидуальному варианту host\_main.cpp}, label=lst1, style=Go]
#include <iostream>
#include <stdio.h>
#include <stdexcept>
#include <iomanip>
#ifdef _WINDOWS
#include <io.h>
#else
#include <unistd.h>
#endif


#include "experimental/xrt_device.h"
#include "experimental/xrt_kernel.h"
#include "experimental/xrt_bo.h"
#include "experimental/xrt_ini.h"

#include "gpc_defs.h"
#include "leonhardx64_xrt.h"
#include "gpc_handlers.h"

#define BURST 256

union uint64 {
    uint64_t 	u64;
    uint32_t 	u32[2];
    uint16_t 	u16[4];   
    uint8_t 	u8[8];   
};

uint64_t rand64() {
    uint64 tmp;
    tmp.u32[0] =  rand();
    tmp.u32[1] =  rand();
    return tmp.u64;
}

static void usage()
{
	std::cout << "usage: <xclbin> <sw_kernel>\n\n";
}

int main(int argc, char** argv)
{

	unsigned int cores_count = 0;
	float LNH_CLOCKS_PER_SEC;

	__foreach_core(group, core) cores_count++;

	if (argc < 3) {
		usage();
		throw std::runtime_error("FAILED_TEST\nNo xclbin specified");
	}

	leonhardx64 lnh_inst = leonhardx64(0,argv[1]);
	__foreach_core(group, core)
	{
		lnh_inst.load_sw_kernel(argv[2], group, core);
	}

	uint64_t *host2gpc_buffer[LNH_GROUPS_COUNT][LNH_MAX_CORES_IN_GROUP];
	__foreach_core(group, core)
	{
		host2gpc_buffer[group][core] = (uint64_t*) malloc(2*BURST*sizeof(uint64_t));
	}
	uint64_t *gpc2host_buffer[LNH_GROUPS_COUNT][LNH_MAX_CORES_IN_GROUP];
	__foreach_core(group, core)
	{
		gpc2host_buffer[group][core] = (uint64_t*) malloc(2*BURST*sizeof(uint64_t));
	}
    
	__foreach_core(group, core)
	{
		for (int i=0;i<BURST;i++)
		{
            		uint64_t a = rand64()%1025;
            		uint64_t b = rand64() %1025;
			host2gpc_buffer[group][core][2*i] = a;
			host2gpc_buffer[group][core][2*i+1] = b;

		}
	}

	__foreach_core(group, core) {
		lnh_inst.gpc[group][core]->start_async(__event__(insert_burst));
	}

	__foreach_core(group, core) {
		lnh_inst.gpc[group][core]->buf_write(BURST*2*sizeof(uint64_t),(char*)host2gpc_buffer[group][core]);
	}

	__foreach_core(group, core) {
		lnh_inst.gpc[group][core]->buf_write_join();
	}

	__foreach_core(group, core) {
		lnh_inst.gpc[group][core]->mq_send(BURST);
	}

	__foreach_core(group, core) {
		lnh_inst.gpc[group][core]->start_async(__event__(search_burst));
	}


	unsigned int count[LNH_GROUPS_COUNT][LNH_MAX_CORES_IN_GROUP];

	__foreach_core(group, core) {
		count[group][core] = lnh_inst.gpc[group][core]->mq_receive();
	}



	__foreach_core(group, core) {
		lnh_inst.gpc[group][core]->buf_read(count[group][core]*2*sizeof(uint64_t),(char*)gpc2host_buffer[group][core]);
	}


	__foreach_core(group, core) {
		lnh_inst.gpc[group][core]->buf_read_join();
	}


	bool error = false;

	__foreach_core(group, core) {
		for (int i=0; i<count[group][core]; i++) {
			uint64_t elem = gpc2host_buffer[group][core][i];
            for(int j = 0; j < BURST;j++)
            {
                uint64_t orig_key = host2gpc_buffer[group][core][2*j+1];
                if (elem == orig_key)
                    error = true;
            }
		}
	}


	__foreach_core(group, core) {
		free(host2gpc_buffer[group][core]);
		free(gpc2host_buffer[group][core]);
	}
	
	

	if (!error)
		printf("???? ??????? ???????!\n");
	else
		printf("???? ???????? ? ???????!\n");


	return 0;
}
\end{lstlisting}
\begin{lstlisting}[caption={Код программы по индивидуальному варианту sw\_kernel\_main.c}, label=lst2, style=Go]

#include <stdlib.h>
#include <unistd.h>
#include "lnh64.h"
#include "gpc_io_swk.h"
#include "gpc_handlers.h"

#define SW_KERNEL_VERSION 26
#define DEFINE_LNH_DRIVER
#define DEFINE_MQ_R2L
#define DEFINE_MQ_L2R
#define __fast_recall__

#define A_STRUCTURE 1
#define B_STRUCTURE 2
#define R_STRUCTURE 3

extern lnh lnh_core;
extern global_memory_io gmio;
volatile unsigned int event_source;

int main(void) {
    /////////////////////////////////////////////////////////
    //                  Main Event Loop
    /////////////////////////////////////////////////////////
    //Leonhard driver structure should be initialised
    lnh_init();
    //Initialise host2gpc and gpc2host queues
    gmio_init(lnh_core.partition.data_partition);
    for (;;) {
        //Wait for event
        while (!gpc_start());
        //Enable RW operations
        set_gpc_state(BUSY);
        //Wait for event
        event_source = gpc_config();
        switch(event_source) {
            /////////////////////////////////////////////
            //  Measure GPN operation frequency
            /////////////////////////////////////////////
            case __event__(insert_burst) : insert_burst(); break;
            case __event__(search_burst) : search_burst(); break;
        }
        //Disable RW operations
        set_gpc_state(IDLE);
        while (gpc_start());

    }
}

//-------------------------------------------------------------
//      ???????? ????? ?? ?????????? ?????? ? ?????? ? lnh64
//-------------------------------------------------------------
 
void insert_burst() {

    lnh_del_str_sync(A_STRUCTURE);
    lnh_del_str_sync(B_STRUCTURE);
    lnh_del_str_sync(R_STRUCTURE);
    unsigned int count = mq_receive();
    unsigned int size_in_bytes = 2*count*sizeof(uint64_t);
    uint64_t *buffer = (uint64_t*)malloc(size_in_bytes);
    buf_read(size_in_bytes, (char*)buffer);
    for (int i=0; i<count; i++) {
        lnh_ins_sync(A_STRUCTURE,buffer[2*i],0);
        lnh_ins_sync(B_STRUCTURE, buffer[2*i+1],0);
    }
    lnh_sync();
    free(buffer);
}


//-------------------------------------------------------------
//      ????? ????????? lnh64 ? ?????? ? ?????????? ?????? 
//-------------------------------------------------------------
 
void search_burst() {

    lnh_sync(); 
    lnh_not_sync(A_STRUCTURE, B_STRUCTURE,R_STRUCTURE);
    unsigned int count = lnh_get_num(R_STRUCTURE);
    unsigned int size_in_bytes = count*sizeof(uint64_t);
    uint64_t *R = (uint64_t*)malloc(size_in_bytes);
    
    for (int i=0; i<count; i++) {
        R[i] = lnh_core.result.key;
        lnh_next(R_STRUCTURE,lnh_core.result.key);
    }
    
    buf_write(size_in_bytes, (char*)R);   
    mq_send(count);
    free(R);

}
\end{lstlisting}
\par В листинге~\ref{lst3} представлена команда для сборки проекта.
\begin{lstlisting}[caption={Код программы по индивидуальному варианту sw\_kernel\_main.c}, label=lst3, style=Go]
./host_main leonhard_2cores_267mhz.xclbin sw_kernel_main.rawbinary
\end{lstlisting}
\newpage