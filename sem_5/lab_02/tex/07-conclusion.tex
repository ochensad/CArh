\section*{Заключение}
\addcontentsline{toc}{section}{Заключение}
В данной лабораторной работе было проведено ознакомление с архитектурой ядра Taiga, а именно с порядком работы вычислительного конвейера: изучены команды RV32I, рассмотрены действия, выполняемые
на каждой стадии конвейера, и данные, передаваемые между ними.

После ознакомления с теоретической стороной вопроса, был выполнен разбор этапов выполнения программы на симуляции процессора с набором инструкций RV32I. После ее анализа были сделаны выводы, что требуется оптимизация. Программу можно было оптимизировать на 20\%.

В итоге, теоретические знания о порядке исполнения программ на процессорах с RISC архитектурой были закреплены на практике. 

Таким образом все поставленные задачи решены, основная цель работы достигнута.
\newpage