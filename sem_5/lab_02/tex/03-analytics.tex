\section{Аналитическая часть}

\subsection{Архитектура набора команд RV32I}
RISC-V является открытым современным набором команд, который может использоваться для построения как микроконтроллеров, так и высокопроизводительных микропроцессоров. В связи с такой широкой областью применения в систему команд введена вариативность. Таким образом, термин RISC-V фактически является названием для семейства различных систем команд, которые строятся вокруг базового набора команд, путем внесения в него различных расширений.

В данной работе исследуется набор команд RV32I, который включает в себя основные команды 32-битной целочисленной арифметики кроме умножения и деления. В рамках данного набора команд мы не будем рассматривать системные команды, связанные с таймерами, системными регистрами, управлением привилегиями, прерываниями и исключениями.

В настоящем разделе описывается архитектура набора команд, то есть архитектура абстрактной вычислительной машины с точки зрения набора команд без связи с конкретной аппаратной реализацией.

\subsection{Микроархитектура}
Теперь перейдем от рассмотрения абстрактной архитектуры системы команд к рассмотрению микроархитектуры ядра Taiga.

Будем рассматривать систему, состоящую из вычислительного ядра Taiga и локальной памяти, реализованной с помощью блочной памяти ПЛИС. Данная память является статической, синхронной и двухпортовой. Один и тот же блок памяти используется для реализации как памяти команд (ПК), так и памяти данных (ПД). Таким образом команды и данные находятся в едином адресном пространстве. Дешифратор адресов настроен таким образом, что блок памяти ПЛИС отображается в адресное пространство RISC-V с адреса 0x80000000, как мы это видели из рассмотрения примера выше.

Благодаря двухпортовой организации имеется возможность чтения и записи одновременно и команд и данных. Кроме того, блочная память ПЛИС имеет фиксированную задержку доступа в 1 такт. Таким образом, в нашей системе не будут возникать задержки доступа к памяти, в связи с чем отпадает необходимость в кеш-памяти.

Taiga является конвейерным микропроцессором с элементами суперскалярности. При конвейерной организации микропроцессора различные команды одновременно проходят различные стадии своей обработки. Конвейер Taiga насчитывает 4 стадии. В скобках приведены сокращенные обозначения стадий.

\begin{enumerate}
	\item выборка(F). Стадия, на которой команда извлекается из ПК. Выполняется в блоке выборки;
	\item диспетчеризация (ID). Стадия, на которой происходит запись команды в очередь команд для декодирования. Выполняется в блоке управления метаданными;
	\item декодирование и планирование на выполнение (D). Стадия на которой происходит определение типа и полей команды и определение вычислительного блока, способного ее исполнить. Выполняется в блоке декодирования и планирования на выполнение;
	\item выполнение (AL, M1..M3, в зависимости от исполнительного блока). Стадия, на которой команда передается в блок выполнения.	
\end{enumerate}

"Ширина" конвейера Taiga (то есть, количество команд, которые одновременно могут находиться на одной и той же стадии конвейера) равна 1 для всех стадий, кроме стадии выполнения. В лучшем случае, каждая стадия конвейера (кроме выполнения) выполняется за один такт.

В состав рассматриваемой конфигурации Taiga входит 3 блока выполнения команд: Арифметико-логическое устройство (АЛУ), блок доступа к памяти (LSU) и блок ветвлений. АЛУ и блок ветвлений выполняют команды за 1 такт, LSU -- минимум за 3. Таким образом, возможна ситуация когда команда обращения к памяти выполняется одновременно с арифметической командой.

На рисунке~\ref{img:taiga_pipeline.png} показана упрощенная и укрупненная структурная схема ядра Taiga.

\img{0.45\textwidth}{taiga_pipeline.png}{Схема ядра Taiga}
\newpage