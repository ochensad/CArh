%\pagenumbering{arabic}
\section*{Введение}
\phantomsection
\addcontentsline{toc}{section}{Введение}
Основной \textbf{целью} работы является ознакомление с принципами функционирования, построения и особенностями архитектуры суперскалярных конвейерных микропроцессоров. Дополнительной целью работы является знакомство с принципами проектирования и верификации сложных цифровых устройств с использованием языка описания аппаратуры SystemVerilog и ПЛИС.

Для достижения поставленной цели необходимо решить следующие \textbf{задачи}:

\begin{itemize}
	\item ознакомиться с набором команд RV32I;
	\item ознакомиться с основными принципами работы ядра Taiga -- изучить операции, выполняемые на каждой стадии обработки команд;
	\item на основе полученных знаний проанализировать ход выполнения программы и оптимизировать ее.
\end{itemize}

\newpage